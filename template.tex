\documentclass[italian, a4paper, 10pt, twocolumn]{../../style/lab_unige}
    \usepackage[a4paper, margin=1.25cm, footskip=0.25in]{geometry}

    \usepackage[utf8]{inputenc}
    \usepackage[T1]{fontenc}

    \usepackage[italian]{babel}

    % \usepackage{biblatex}

    \usepackage[bookmarksopen=true, 
    citebordercolor={0 1 0}, 
    linkbordercolor={1 0 0}, 
    urlbordercolor={0 1 1}]{hyperref}
    \usepackage[numbered]{bookmark}

    \usepackage{graphicx}
    \graphicspath{{../fig/}}
    \usepackage{array}
    \usepackage{tabulary}
    \usepackage{booktabs}

    % FOUNDAMENTAL
    \usepackage{../../style/custom}

    \usepackage{physics}

    \usepackage{breqn}
    \usepackage{cuted}
    \usepackage{txfonts}

    \usepackage{lipsum}

    %% Define ref types
    \newcommand{\reftab}[1]{Tab. {\ref{#1}}}%
    \newcommand{\reffig}[1]{Fig. {\ref{#1}}}%
    \newcommand{\refeqn}[1]{({\ref{#1})}}%
    %%
    \setlength{\columnsep}{6mm}
    \begin{document}


    \twocolumn[
    \begin{@twocolumnfalse}
        
        \title{
            %%TITLE_HERE%%
        }
        \author{
        Eugenio Dormicchi\textsuperscript{1},  
        Riccardo Pizzimbone\textsuperscript{1}, 
        Mattia Sotgia\textsuperscript{1}
        }

        \date{
        \textsuperscript{1}Gruppo C03, Esperienza di laboratorio n. %%NN%% \\
        %\textsuperscript{2}In presenza in laboratorio per la presa dati\\
            % Università degli Studi di Genova, Dipartimento di Fisica.\\
            Presa dati–– 
            %%DATE_HERE%%, 15:00– 18:00; Analisi dati–– 
            <end-date here>
        }
        \maketitle
        
        \begin{abstract}
        
        \textit{Obiettivo– }
        
        \textit{Metodi– }
        
        \textit{Risultati– }
        
        \textit{Conclusione– }
        
        
        \end{abstract}
        \vspace{2em}
        \end{@twocolumnfalse}
    ]

    %%%% CORPO DEL TESTO
    %%%% CORPO DEL TESTO

    \section{Obiettivo}
    \label{section:aim}

    \section{Strumentazione}
    \label{section:strument}

    \section{Metodi}
    \label{section:methods}

    \section{Risultati}
    \label{section:results}

    \section{Conclusione}
    \label{section:conclusion}

    \subsection{Controlli}

    \subsection{Possibili errori sistematici}
    

    \end{document}
    
