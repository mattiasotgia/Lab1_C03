\begin{table}
    \footnotesize
    \centering
    \caption{Tensione al variare della temperatura. Sono stati presi quattro valori della tensione a 30~s, 1~min, 2~min e 3~min dal momento in cui la temperatura del liquido era stabile, poiché per alcuni fenomeni di isteresi dell'elastomero la contrazione presenta un effetto di deriva, e quindi risulta importante effettuare una misura rispetto al tempo. Nel prendere il valore della tensione riportiamo tra parentesi le cifre che scalavano più velocemente, che consideriamo come incertezza massima. Le ultime tre righe riportano valori di tensione a temperature in salita, però osserviamo che le tre temperature (segnate con *) risultano essere estremamente instabili, e tendono a decrescere molto velocemente, perdendo circa 4-6~K in 3~min.}
    \label{table:p2}
    \begin{tabular}{ccccc}
        \hline\hline\\[-1.5ex]
        \multicolumn{4}{c}{Tensione elettrica $V_0$ [mV]} & Temperatura $T_i^{(k)}$ [K] \\[+0.5ex]
        a 30~s  & a 1~min & a 2~min & a 3~min             & $273.15+T_i^{\text{(deg)}}$ \\[+0.5ex] \hline \\[-1.5ex]
        7.7(50) & 7.7(21) & 7.7(00) & 7.6(61)             & $273.15+58$                 \\[+0.5ex]
        7.5(93) & 7.5(85) & 7.5(63) & 7.54(6)             & $273.15+54$                 \\[+0.5ex]
        7.4(95) & 7.4(87) & 7.47(3) & 7.46(0)             & $273.15+50$                 \\[+0.5ex]
        7.3(88) & 7.38(3) & 7.37(4) & 7.36(3)             & $273.15+45$                 \\[+0.5ex]
        7.30(4) & 7.30(1) & 7.29(5) & 7.28(8)             & $273.15+40$                 \\[+0.5ex]
        7.21(3) & 7.21(0) & 7.20(6) & 7.20(0)             & $273.15+35$                 \\[+0.5ex]
        7.14(3) & 7.14(0) & 7.14(0) & 7.13(5)             & $273.15+30$                 \\[+0.5ex]
        7.03(3) & 7.03(1) & 7.03(1) & 7.02(8)             & $273.15+25$                 \\[+0.5ex]
        6.97(6) & 6.97(6) & 6.97(6) & 6.97(7)             & $273.15+20$                 \\[+0.5ex] \hline \\[-1.5ex]
        7.10(6) & 7.10(4) & 7.10(0) & 7.09(6)             & $273.15+30^*$               \\[+0.5ex]
        7.22(1) & 7.21(9) & 7.21(5) & 7.21(1)             & $273.15+39^*$               \\[+0.5ex]
        7.37(8) & 7.37(5) & 7.36(5) & 7.35(8)             & $273.15+49^*$               \\[+0.5ex] \hline \\[-1.5ex]
    \end{tabular}
\end{table}
\begin{table}
    \centering
    \footnotesize
    \caption{Valori di tensione $f$ dell'elastico rispetto alla temperatura $T$.}
    \label{table:p2a}
    \setlength{\tabcolsep}{0.4\tabcolsep}
    \begin{tabular}{ccccc}
        \hline\hline\\[-1.5ex]
        \multicolumn{4}{c}{Tensione $f$ [N]}                              & Temperatura [K] \\[+0.5ex]
        a 30~s  & a 1~min & a 2~min & a 3~min                             & $T_i^{(k)}$     \\[+0.5ex] \hline \\[-1.5ex]
        $2701.5\pm1.0$ & $2691.4\pm1.0$ & $2684.1\pm1.0$ & $2670.5\pm1.0$ & $331.2\pm0.6$   \\[+0.5ex]
        $2646.8\pm1.0$ & $2644.0\pm1.0$ & $2636.3\pm1.0$ & $2630.4\pm0.1$ & $327.2\pm0.6$   \\[+0.5ex]
        $2612.6\pm1.0$ & $2609.9\pm1.0$ & $2605.0\pm0.1$ & $2600.4\pm0.1$ & $323.2\pm0.6$   \\[+0.5ex]
        $2575.3\pm1.0$ & $2573.6\pm0.1$ & $2570.5\pm0.1$ & $2566.6\pm0.1$ & $318.2\pm0.6$   \\[+0.5ex]
        $2546.1\pm0.1$ & $2545.0\pm0.1$ & $2542.9\pm0.1$ & $2540.5\pm0.1$ & $313.2\pm0.6$   \\[+0.5ex]
        $2514.3\pm0.1$ & $2513.3\pm0.1$ & $2511.9\pm0.1$ & $2509.8\pm0.1$ & $308.2\pm0.6$   \\[+0.5ex]
        $2489.9\pm0.1$ & $2488.9\pm0.1$ & $2488.9\pm0.1$ & $2487.1\pm0.1$ & $303.2\pm0.6$   \\[+0.5ex]
        $2451.6\pm0.1$ & $2450.9\pm0.1$ & $2450.9\pm0.1$ & $2449.8\pm0.1$ & $298.2\pm0.6$   \\[+0.5ex]
        $2431.7\pm0.1$ & $2431.7\pm0.1$ & $2431.7\pm0.1$ & $2432.1\pm0.1$ & $293.2\pm0.6$   \\[+0.5ex] \hline \\[-1.5ex]
        $2477.0\pm0.1$ & $2476.3\pm0.1$ & $2474.9\pm0.1$ & $2473.5\pm0.1$ & $303.2\pm0.6$   \\[+0.5ex]
        $2517.1\pm0.1$ & $2516.4\pm0.1$ & $2515.0\pm0.1$ & $2513.6\pm0.1$ & $312.2\pm0.6$   \\[+0.5ex]
        $2571.9\pm0.1$ & $2570.8\pm0.1$ & $2567.3\pm0.1$ & $2564.9\pm0.1$ & $322.2\pm0.6$   \\[+0.5ex] \hline \\[-1.5ex]
    \end{tabular}
\end{table}