%% Document created 05 March 2021 automatically 
%% from /Users/massimosotgia/Desktop/uni_at_DIFI/Lab_C03/setup.py 

%% Copyright (C) Mattia Sotgia et al. 2021
%% Using class lab_unige.cls
%                                                            
%                                                            
%   **                 **             ******   ****   ****   
%  /**                /**            **////** *///** */// *  
%  /**        ******  /**           **    // /*  */*/    /*  
%  /**       ´´´´´´** /******      /**       /* * /*   ***   
%  /**        ******* /**///**     /**       /**  /*  /// *  
%  /**       **´´´´** /**  /** **  //**    **/*   /* *   /*  
%  /********//********/****** /**   //****** / **** / ****   
%  ////////  //////// /////   //     //////   ////   /´///   
%                                                            
%                                                            
\documentclass[italian, a4paper, 10pt, twocolumn]{../../style/lab_unige}
\usepackage[a4paper, margin=1.25cm, footskip=0.25in]{geometry}

\usepackage[utf8]{inputenc}
\usepackage[T1]{fontenc}

\usepackage[italian]{babel}

% \usepackage{biblatex}

\usepackage[bookmarksopen=true, 
citebordercolor={0 1 0}, 
linkbordercolor={1 0 0}, 
urlbordercolor={0 1 1}]{hyperref}
\usepackage[numbered]{bookmark}

\usepackage{graphicx}
\graphicspath{{../fig/}}
\usepackage{array}
\usepackage{tabulary}
\usepackage{booktabs}

% FOUNDAMENTAL
\usepackage{../../style/custom}

\usepackage{physics}

\usepackage{breqn}
\usepackage{cuted}
\usepackage{txfonts}

\usepackage{lipsum}

%% Define ref types
\newcommand{\reftab}[1]{Tab. {\ref{#1}}}%
\newcommand{\reffig}[1]{Fig. {\ref{#1}}}%
\newcommand{\refeqn}[1]{({\ref{#1}})}%
%% PAPER ONLY custom Macros
\newcommand{\gLab}{$g_t=(9.8056\pm0.0001_{\text{stat}}) \text{ m/s}^2$\space}
\newcommand{\ks}{$k_{\text{\small statico}}$\space}
\newcommand{\kd}{$k_{\text{\small dinamico}}$\space}
\newcommand{\ChiSqr}{$\chi^2$\space}
\newcommand{\ChiNdf}{$\chi^2/\text{ndf}$\space}
\newcommand{\cernroot}{\verb|root|\space}
\newcommand{\scidavis}{\verb|scidavis|\space}
\newcommand{\treSigma}{$3\sigma$\space}
\newcommand{\stdNG}[2]{$S_{#1}$($#2$)} %<- ???
\newcommand{\stdErr}[1]{$\varepsilon_{#1}$}
\newcommand{\hookeLaw}{$F=k\cdot\Delta L$\space}
\newcommand{\misuraIncertezaUM}[3]{$#1\pm#2$ #3}
\newcommand{\Lo}{$L_0$\space}
\newcommand{\Li}[1]{$L_{#1}$}
\newcommand{\Ti}[1]{$T_{#1}$}
\newcommand{\MassI}[1]{$m_{#1}$}


%%
\setlength{\columnsep}{6mm}

\begin{document}
    \twocolumn[
    \begin{@twocolumnfalse}
        \title{
            Verifica Sperimentale della Legge di Hooke e Confronto con Modello Statico e Dinamico
        }
        \author{
        Eugenio Dormicchi\textsuperscript{1},
        % Riccardo Pizzimbone\textsuperscript{1}, 
        Giovanni Oliveri\textsuperscript{1},
        Mattia Sotgia\textsuperscript{1, 2}
        }

        \date{
        \textsuperscript{1}Gruppo C03, Esperienza di laboratorio n. 5 \\
        \textsuperscript{2}In presenza in laboratorio per la presa dati\\
            % Università degli Studi di Genova, Dipartimento di Fisica.\\
            Presa dati-- 
            10 Marzo 2021, 15:00– 18:00; Analisi dati-- 
            <end-date here>
        }
        \maketitle
        
        \begin{abstract}
            \textit{Obiettivo--}
            Vogliamo verificare la validità della legge di Hooke per cui la forza $F$ applicata su un corpo elastico
            è direttamente proporzionale all'elongazione causata, secondo la legge \hookeLaw.
            \textit{Metodi--}
            Sfruttiamo due modelli per ricavare in modo differente la costante $k$ legata alla molla. Considerando 
            la molla in una condizione statica, con un corpo di massa nota \MassI{i}, e misurando l'allungamento 
            \Li{i} causato dalla massa, possiamo ricavare \ks. Se invece mettiamo in oscillazione dalla condizione 
            di equilibrio \Lo possiamo dal periodo \Ti{i} ricavare \kd (considerando il moto nel regime delle piccole
            oscillazioni).
            \textit{Risultati--}
        
            \textit{Conclusione--}
        
        
        \end{abstract}
        \vspace{2em}
    \end{@twocolumnfalse}
    ]

    %%%% CORPO DEL TESTO
    %%%% CORPO DEL TESTO

    \section{Obiettivo}
    \label{section:aim}

    \section{Strumentazione}
    \label{section:strument}

    \section{Metodi}
    \label{section:methods}

    \section{Risultati}
    \label{section:results}

    \section{Conclusione}
    \label{section:conclusion}

    \subsection{Controlli}

    \subsection{Possibili errori sistematici}
    

\end{document}
    
