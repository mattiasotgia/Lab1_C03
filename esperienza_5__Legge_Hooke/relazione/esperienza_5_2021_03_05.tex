%% Document created 05 March 2021 automatically 
%% from /Users/massimosotgia/Desktop/uni_at_DIFI/Lab_C03/setup.py 

%% Copyright (C) Mattia Sotgia et al. 2021
%% Using class lab_unige.cls
%                                                            
%                                                            
%   **                 **             ******   ****   ****   
%  /**                /**            **////** *///** */// *  
%  /**        ******  /**           **    // /*  */*/    /*  
%  /**       ´´´´´´** /******      /**       /* * /*   ***   
%  /**        ******* /**///**     /**       /**  /*  /// *  
%  /**       **´´´´** /**  /** **  //**    **/*   /* *   /*  
%  /********//********/****** /**   //****** / **** / ****   
%  ////////  //////// /////   //     //////   ////   /´///   
%                                                            
%                                                            
\documentclass[italian, a4paper, 10pt, twocolumn]{../../style/lab_unige}
\usepackage[a4paper, margin=1.25cm, footskip=0.25in]{geometry}

\usepackage[utf8]{inputenc}
\usepackage[T1]{fontenc}

\usepackage[italian]{babel}

% \usepackage{biblatex}

\usepackage[bookmarksopen=true, 
citebordercolor={0 1 0}, 
linkbordercolor={1 0 0}, 
urlbordercolor={0 1 1}]{hyperref}
\usepackage[numbered]{bookmark}

\usepackage{graphicx}
\graphicspath{{../fig/}}
\usepackage{array}
\usepackage{tabulary}
\usepackage{booktabs}

% FOUNDAMENTAL
\usepackage{../../style/custom}

\usepackage{physics}

\usepackage{breqn}
\usepackage{cuted}
\usepackage{txfonts}

\usepackage{lipsum}

%% Define ref types
\newcommand{\reftab}[1]{Tab. {\ref{#1}}}%
\newcommand{\reffig}[1]{Fig. {\ref{#1}}}%
\newcommand{\refeqn}[1]{({\ref{#1}})}%
%% PAPER ONLY custom Macros
\newcommand{\g_lab}{$g_{t}=(9.8056\pm0.0001_{\text{stat}}) \text{ m/s}^{2}$}
\newcommand{\ks}{$k_{\text{statico}}$}
\newcommand{\kd}{$k_{\text{dinamico}}$}
\newcommand{\chi_sqr}{$\chi^2$}
\newcommand{\chi_ndf}{$\chi^2/\text{ndf}$}
\newcommand{\root}{\verb|root|}
\newcommand{\scidavis}{\verb|scidavis|}
\newcommand{\treSigma}{$3\sigma$}
% \newcommand{\stdN}[2]{$S_{#1}$($#2$)} <- ???
\newcommand{\hookeLaw}{$F=k\cdot\Delta L$}
\newcommand{\Lo}{$L_0$}


%%
\setlength{\columnsep}{6mm}

\begin{document}
    \twocolumn[
    \begin{@twocolumnfalse}
        \title{
            Verifica Sperimentale della Legge di Hooke su un Modello Dinamico e un Modello Statico
        }
        \author{
        Eugenio Dormicchi\textsuperscript{1},
        % Riccardo Pizzimbone\textsuperscript{1}, 
        Giovanni Oliveri\textsuperscript{1},
        Mattia Sotgia\textsuperscript{1, 2}
        }

        \date{
        \textsuperscript{1}Gruppo C03, Esperienza di laboratorio n. 5 \\
        \textsuperscript{2}In presenza in laboratorio per la presa dati\\
            % Università degli Studi di Genova, Dipartimento di Fisica.\\
            Presa dati-- 
            10 March 2021, 15:00– 18:00; Analisi dati-- 
            <end-date here>
        }
        \maketitle
        
        \begin{abstract}
            \textit{Obiettivo-- }
        
            \textit{Metodi-- }
        
            \textit{Risultati-- }
        
            \textit{Conclusione-- }
        
        
        \end{abstract}
        \vspace{2em}
    \end{@twocolumnfalse}
    ]

    %%%% CORPO DEL TESTO
    %%%% CORPO DEL TESTO

    \section{Obiettivo}
    \label{section:aim}

    \section{Strumentazione}
    \label{section:strument}

    \section{Metodi}
    \label{section:methods}

    \section{Risultati}
    \label{section:results}

    \section{Conclusione}
    \label{section:conclusion}

    \subsection{Controlli}

    \subsection{Possibili errori sistematici}
    

\end{document}
    
