\begin{table}[t]
  \centering
  \footnotesize
  \caption{Valori delle masse e dei relativi allungamenti $\delta l$. L'errore sull'allungamento $\delta l$ è 
  preso considerando la minima variazione misurabile dal foglio di carta millimetrata. Per rendere statistica 
  la misura si è utilizzata la regola del \treSigma e si è così ottenuto il valore di 
  $\varepsilon_{\delta l} = \Delta l / \sqrt{3}$.}
  \label{table:sts_values}
  \begin{tabular}{lcc}
      \hline\hline\\[-1.5ex]
        & Massa                      & Elongazione                             \\[+0.5ex]
        & $m_i\pm\varepsilon_m$ (kg) & $\delta l\pm\varepsilon_{\delta l}$ (m) \\[+0.5ex] \hline \\[-1.5ex]
      1 & $0.261499\pm0.000002$      & $0.0380\pm0.0006$                       \\[+0.5ex]
      2 & $0.650447\pm0.000007$      & $0.0900\pm0.0006$                       \\[+0.5ex]
      3 & $0.796669\pm0.000007$      & $0.1120\pm0.0006$                       \\[+0.5ex]
      4 & $1.035634\pm0.000009$      & $0.1440\pm0.0006$                       \\[+0.5ex]
      5 & $0.393091\pm0.000005$      & $0.0550\pm0.0006$                       \\[+0.5ex]
      6 & $0.916857\pm0.000009$      & $0.1280\pm0.0006$                       \\[+0.5ex]
      \hline \\[-1.5ex]
  \end{tabular}
\end{table}

\begin{table}[t]
    \centering
    \footnotesize
    \caption{Valori delle masse e relativi valori di periodo $\bar{T_i}$. Sono riportati anche i valori di
    $\bar{T_i^2}$. Gli errori relativi ai periodi sono ricavati dal calcolo dell'errore standard ($\varepsilon$). 
    L'errore sulla massa preso dalla linearità dello strumento (0.004 g) è ottenuto dalla somma degli errori dovuti 
    alle diverse pesate. Il valore è poi staticizzato per la regola del \treSigma ($\varepsilon_m = \Delta m/\sqrt{3}$)}
    \label{table:dyn_values}
    \begin{tabular}{lccc}
        \hline\hline\\[-1.5ex]
          & Massa                      & Periodo                         & Periodo al quadrato                       \\[+0.5ex]
          & $m_i\pm\varepsilon_m$ (kg) & $\bar{T_i}\pm\varepsilon_T$ (s) & $\bar{T_i^2}\pm\varepsilon_{T^2}$ (s$^2$) \\[+0.5ex] \hline \\[-1.5ex]
        1 & $0.261499\pm0.000002$      & $0.3761\pm0.0015$               & $0.1415\pm0.0011$                         \\[+0.5ex]
        2 & $0.650447\pm0.000007$      & $0.5646\pm0.0020$               & $0.3188\pm0.0022$                         \\[+0.5ex]
        3 & $0.796669\pm0.000007$      & $0.6217\pm0.0018$               & $0.3865\pm0.0022$                         \\[+0.5ex]
        4 & $1.035634\pm0.000009$      & $0.6962\pm0.0011$               & $0.4847\pm0.0015$                         \\[+0.5ex]
        5 & $0.393091\pm0.000005$      & $0.439 \pm0.003 $               & $0.1925\pm0.0027$                         \\[+0.5ex]
        6 & $0.916857\pm0.000009$      & $0.6574\pm0.0017$               & $0.4322\pm0.0021$                         \\[+0.5ex]
        \hline \\[-1.5ex]
        
    \end{tabular}
\end{table}

