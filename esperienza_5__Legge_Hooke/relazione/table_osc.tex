\begin{table}[t]
    \centering
    \footnotesize
    \caption{Valori delle masse e relativi valori di periodo $\bar{T_i}$. Sono riportati anche i valori di
    $\bar{T_i^2}$. Gli errori relativi ai periodi sono ricavati dal calcolo dell'errore standard ($\varepsilon$). 
    L'errore sulla massa preso dalla linearità dello strumento (0.004 g) è ottenuto dalla somma degli errori dovuti 
    alle diverse pesate. Il valore è poi staticizzato per la regola del \treSigma ($\varepsilon_m = \Delta m/\sqrt{3}$)}
    \label{table:dyn_values}
    \begin{tabular}{lccc}
        \hline\hline\\[-1.5ex]
          & $m_i\pm\varepsilon_m$ (kg) & Periodo $\bar{T_i}\pm\varepsilon_T$ (s) & Periodo al quadrato $\bar{T_i^2}\pm\varepsilon_{T^2}$ (s$^2$) \\[+0.5ex] \hline \\[-1.5ex]
        1 & 0.261499$\pm$2e-06         & $0.3761\pm0.0015$                       & $0.1415\pm0.0011$                                             \\[+0.5ex]
        2 & 0.650447$\pm$7e-06         & $0.5646\pm0.0020$                       & $0.3188\pm0.0022$                                             \\[+0.5ex]
        3 & 0.796669$\pm$7e-06         & $0.6217\pm0.0018$                       & $0.3865\pm0.0022$                                             \\[+0.5ex]
        4 & 1.035634$\pm$9e-06         & $0.6962\pm0.0011$                       & $0.4847\pm0.0015$                                             \\[+0.5ex]
        5 & 0.393091$\pm$5e-06         & $0.439 \pm0.003 $                       & $0.1925\pm0.0027$                                             \\[+0.5ex]
        6 & 0.916857$\pm$9e-06         & $0.6574\pm0.0017$                       & $0.4322\pm0.0021$                                             \\[+0.5ex]
        \hline \\[-1.5ex]
        
    \end{tabular}
\end{table}

\begin{table}[t]
    \centering
    \footnotesize
    \caption{Valori delle masse e dei relativi allungamenti $\delta l$. L'errore sull'allungamento $\delta l$ è 
    preso considerando la minima variazione misurabile dal foglio di carta millimetrata. Per rendere statistica 
    la misura si è utilizzata la regola del \treSigma e si è così ottenuto il valore di 
    $\varepsilon_{\delta l} = \Delta l / \sqrt{3}$.}
    \label{table:sts_values}
    \begin{tabular}{}
        \hline\hline\\[-1.5ex]
          & $m_i\pm\varepsilon_m$ (kg) & Allungamento $\delta l\pm\varepsilon_{\delta l}$ (m) \\[+0.5ex] \hline \\[-1.5ex]
        1 & 0.261499$\pm$2e-06         &                                                      \\[+0.5ex]
        2 & 0.650447$\pm$7e-06         &                                                      \\[+0.5ex]
        3 & 0.796669$\pm$7e-06         &                                                      \\[+0.5ex]
        4 & 1.035634$\pm$9e-06         &                                                      \\[+0.5ex]
        5 & 0.393091$\pm$5e-06         &                                                      \\[+0.5ex]
        6 & 0.916857$\pm$9e-06         &                                                      \\[+0.5ex]
        
    \end{tabular}
\end{table}